\documentclass[utf8, russian, hpadding=5mm, vpadding=15mm, floatsection, columnxxvi, columnxxxi, columnxxxii, equationsection, pointsection, footnoteasterisk]{eskdtext}

\usepackage{ucs}							%хз
\usepackage{cmap}						%решение трабла с крокозябрами в pdf

\usepackage[normalem]{ulem}

\usepackage{setspace}
\usepackage{tocloft}
\usepackage[dvipsnames]{xcolor}
\usepackage{indentfirst}

\usepackage{nameref}
\usepackage{hyperref}
\hypersetup{
    linktoc=all,
    pdftex,
    unicode=true,
    colorlinks=true,
    linkcolor=blue,
    citecolor=red,
    urlcolor=green,
    %linktocpage=true,
    pdfhighlight=/N
}

% подключение дополнительных пакетов
\usepackage{amsfonts, amsmath, amssymb, mathtext}
\newcommand{\No}{\textnumero}

\graphicspath{{./pics/}{./images/}{./pictures/}}
\usepackage{wallpaper}

\renewcommand{\familydefault}{\sfdefault}

\newcommand{\squpp}{\vspace{-4mm}}
\newcommand{\squp}{\vspace{-2mm}}
\newcommand{\sqdownn}{\vspace{2mm}}
\newcommand{\sqdown}{\vspace{1mm}}

\newcommand{\squeezeup}{\vspace{-4mm}}
\newcommand{\squeezeeup}{\vspace{-2mm}}
\newcommand{\squeezeedown}{\vspace{2mm}}
\newcommand{\squeezeeedown}{\vspace{1mm}}

%\renewcommand{\cftaftertoctitle}{\hfill}
\renewcommand{\cfttoctitlefont}{\hspace{7cm}\Large\bfseries}	%KOSTIL'
%\renewcommand{\cftaftertoctitle}{\hfill}
\renewcommand{\cftdot}{.}
\renewcommand{\cftsecleader}{\cftdotfill{\cftdotsep}} % for sections
\makeatletter
\g@addto@macro\cftsecfont{\bfseries}
\g@addto@macro\cftsubsecfont{\bfseries}
\g@addto@macro\cftsubsubsecfont{\bfseries}
\g@addto@macro\cftparafont{\bfseries}
\g@addto@macro\cftsubparafont{\bfseries}
\g@addto@macro\cftsubparafont{\bfseries}
\g@addto@macro\cftfigfont{\bfseries}
\g@addto@macro\cfttabfont{\bfseries}

\g@addto@macro\cftsecpagefont{\bfseries}
\g@addto@macro\cftsubsecpagefont{\bfseries}
\g@addto@macro\cftsubsubsecpagefont{\bfseries}
\g@addto@macro\cftparapagefont{\bfseries} 
\g@addto@macro\cftsubparapagefont{\bfseries}
\g@addto@macro\cftfigpagefont{\bfseries}
\g@addto@macro\cfttabpagefont{\bfseries}
\makeatother


% стиль оформления ссылок на источники
\bibliographystyle{ugost2008}

% Для определения форматирования нумерации элементов списка литературы
\makeatletter
\renewcommand{\@biblabel}[1]{#1}
\makeatother

\ESKDtitle{{\Large Название работы}\\ \small Пояснительная записка}
\ESKDsignature{КСУИ.102.4135.001 ПЗ}
\ESKDgroup{\footnotesize Университет ИТМО\\Кафедра СУиИ\\гр.~R4235}
\ESKDauthor{\resizebox{2.22cm}{\height}{Ватрушкин Н. Ю.}}
%\ESKDchecker{}
%\ESKDnormContr{\resizebox{2.22cm}{\height}{}}
%\ESKDapprovedBy{}

% оступы от заголовков разделов и подразделов
\ESKDsectSkip{section}{7mm}{7mm}
\ESKDsectSkip{subsection}{5mm}{5mm}
\ESKDsectSkip{subsubsection}{3mm}{3mm}


\begin{document}
    \addtocounter{page}{0}
    %\input{title_pages}
    \ESKDthisStyle{formII}
    \renewcommand{\cftaftertoctitle}{\thispagestyle{empty}}

	\begingroup
		\hypersetup{linkcolor=black}
		\tableofcontents
	\endgroup    
    
    \newpage
    \topmargin = 0 mm
    
    % ---------------------------------    
    	%\cleardoublepage
	\phantomsection
	\addcontentsline{toc}{section}{Введение}
	\section*{Введение}
Текст введения.
\newpage
    \input{section_1}
    \input{conclusion}
    
    \renewcommand\refname{Список использованных источников}
        \begin{thebibliography}{99}
    	%	\addcontentsline{toc}{section}{Список использованных источников}
    	%	\vspace{-1cm}
    	%{\small
    	%\hypertarget{Yang}{}
    	\bibitem{HRI} The Encyclopedia of Human-Computer Interaction / Mads Soegaard, Rikke Friis Dam --- The Interaction Design Foundation, 2nd Ed.
    	
    	\bibitem{Folding} Adrian A. Canutescu, Roland L. Dunbrack, Cyclic coordinate descent: a robotics algorithm for protein loop closure, Protein Science 12 (5) (2003) 963–972.
    	
    	\bibitem{Buss} Buss S. R. Introduction to inverse kinematics with jacobian transpose, pseudoinverse and damped least squares methods //IEEE Journal of Robotics and Automation. – 2004. – Т. 17. – No. 1-19. – С. 16.
    	
    	\bibitem{Balestrino} A. Balestrino, G. De Maria, and L. Sciavicco. Robust control of robotic manipulators. In Proceed-
    	ings of the 9th IFAC World Congress, volume 5, pages 2435–2440, 1984.
    	
    	\bibitem{Wolovich} W. A. Wolovich and H. Elliott. A computational technique for inverse kinematics. The 23rd IEEE
    	Conference on Decision and Control, 23:1359–1363, December 1984.
    	
    	\bibitem{Buss2} Samuel R. Buss and Jin-Su Kim. Selectively damped least squares for inverse kinematics. Journal of Graphics Tools, 10(3):37–49, 2005.
    	
    	\bibitem{Pechev} Alexandre N. Pechev. Inverse kinematics without matrix invertion. In Proceedings of the 2008 IEEE International Conference on Robotics and Automation, pages 2005–2012, Pasadena, CA, USA, May 19-23 2008.
    	
    	
    	\bibitem{Fletcher} R. Fletcher. Practical methods of optimization; (2nd Ed.). Wiley-Interscience, New York, NY, USA, 1987.
    	
    	\bibitem{Kwan} Kwan W. Chin, B. R. von Konsky, and A. Marriott. Closed-form and generalized inverse kinematics solutions for the analysis of human motion. volume 5, pages 1911–1914, 1997.
    	
    	\bibitem{Steven} Steven M. LaValle. Planning Algorithms. Cambridge University Press, New York, NY, USA, 2006.
    	
    	\bibitem{Li} Li-Chun Tommy Wang and Chih Cheng Chen. A combined optimization method for solving the inverse kinematics problems of mechanical manipulators. IEEE Transactions on Robotics and Automation, 7(4):489–499, 1991.
    	
    	\bibitem{Aristidou} Andreas Aristidou and Joan Lasenby. FABRIK: a fast, iterative solver for the inverse kinematics
    	problem. Submitted to Graphical Models, 2010.
    	%}
    	
    \end{thebibliography} 


    \input{appendix_1}
    
\end{document}